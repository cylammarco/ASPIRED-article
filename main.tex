% mnras_template.tex 
%
% LaTeX template for creating an MNRAS paper
%
% v3.0 released 14 May 2015
% (version numbers match those of mnras.cls)
%
% Copyright (C) Royal Astronomical Society 2015
% Authors:
% Keith T. Smith (Royal Astronomical Society)

% Change log
%
% v3.0 May 2015
%    Renamed to match the new package name
%    Version number matches mnras.cls
%    A few minor tweaks to wording
% v1.0 September 2013
%    Beta testing only - never publicly released
%    First version: a simple (ish) template for creating an MNRAS paper

%%%%%%%%%%%%%%%%%%%%%%%%%%%%%%%%%%%%%%%%%%%%%%%%%%
% Basic setup. Most papers should leave these options alone.
\documentclass[fleqn,usenatbib]{mnras}

% MNRAS is set in Times font. If you don't have this installed (most LaTeX
% installations will be fine) or prefer the old Computer Modern fonts, comment
% out the following line
\usepackage{newtxtext,newtxmath}
% Depending on your LaTeX fonts installation, you might get better results with one of these:
%\usepackage{mathptmx}
%\usepackage{txfonts}

% Use vector fonts, so it zooms properly in on-screen viewing software
% Don't change these lines unless you know what you are doing
\usepackage[T1]{fontenc}

% Allow "Thomas van Noord" and "Simon de Laguarde" and alike to be sorted by "N" and "L" etc. in the bibliography.
% Write the name in the bibliography as "\VAN{Noord}{Van}{van} Noord, Thomas"
\DeclareRobustCommand{\VAN}[3]{#2}
\let\VANthebibliography\thebibliography
\def\thebibliography{\DeclareRobustCommand{\VAN}[3]{##3}\VANthebibliography}


%%%%% AUTHORS - PLACE YOUR OWN PACKAGES HERE %%%%%

% Only include extra packages if you really need them. Common packages are:
\usepackage{graphicx}	% Including figure files
\usepackage{amsmath}	% Advanced maths commands
\usepackage{amssymb}	% Extra maths symbols

%%%%%%%%%%%%%%%%%%%%%%%%%%%%%%%%%%%%%%%%%%%%%%%%%%

%%%%% AUTHORS - PLACE YOUR OWN COMMANDS HERE %%%%%

% Please keep new commands to a minimum, and use \newcommand not \def to avoid
% overwriting existing commands. Example:
%\newcommand{\pcm}{\,cm$^{-2}$}	% per cm-squared

%%%%%%%%%%%%%%%%%%%%%%%%%%%%%%%%%%%%%%%%%%%%%%%%%%

%%%%%%%%%%%%%%%%%%% TITLE PAGE %%%%%%%%%%%%%%%%%%%

% Title of the paper, and the short title which is used in the headers.
% Keep the title short and informative.
\title[ASPIRED]{Automate your spectral data reduction with Automated SpectroPhotometric Image REDuction (ASPIRED) pipeline builder}

% The list of authors, and the short list which is used in the headers.
% If you need two or more lines of authors, add an extra line using \newauthor
\author[M. C. Lam et al.]{
Marco C. Lam,$^{1, 2, 3}$\thanks{E-mail: lam@tau.ac.il}
Robert J. Smith,$^{2}$
Josh Veitch-Michaelis,$^{2}$
P. Ross McWhirter,$^{2}$
Iain A. Steele,$^{2}$
\newauthor
Iair Arcavi,$^{1}$
Lukasz Wyrzykowski$^{3}$
\\
% List of institutions
$^{1}$Tel Aviv University\\
$^{2}$Liverpool John Moores University\\
$^{3}$Warsaw University
}

% These dates will be filled out by the publisher
\date{Accepted XXX. Received YYY; in original form ZZZ}

% Enter the current year, for the copyright statements etc.
\pubyear{2021}

% Don't change these lines
\begin{document}
\label{firstpage}
\pagerange{\pageref{firstpage}--\pageref{lastpage}}
\maketitle

% Abstract of the paper
\begin{abstract}
We provide a suite of publicly available spectral data reduction software
to facilitate rapid scientific outcomes from time-domain observations. The Automated
SpectroPhotometric REDuction (\textsc{ASPIRED}) pipeline builder, designed for common
use on different instruments. The default settings support many typical long-slit
spectrometer configurations, whilst it also offers a flexible set of functions for
users to refine and tailor-make their automated pipelines to an instrument's
individual characteristics. Such automation provides near real-time data reduction
to allow adaptive observing strategies, which is particularly important in the Time
Domain Astronomy.
\end{abstract}

% Select between one and six entries from the list of approved keywords.
% Don't make up new ones.
\begin{keywords}
keyword1 -- keyword2 -- keyword3
\end{keywords}

%%%%%%%%%%%%%%%%%%%%%%%%%%%%%%%%%%%%%%%%%%%%%%%%%%

%%%%%%%%%%%%%%%%% BODY OF PAPER %%%%%%%%%%%%%%%%%%

\section{Introduction}
With major global investments in multi-wavelength and multi-messenger surveys, time domain
astronomy is entering a golden age. To maximally scientific exploit discoveries from these
facilities rapid spectroscopic follow-up observations of transient objects~(e.g.,\ supernovae,
gravitational wave optical counterparts etc.) will provide crucial {\em astrophysical} 
interpretations. Part of the OPTICON\footnote{\url{https://www.astro-opticon.org/}} project
coordinates the operation of a network of largely self-funded European robotic and conventional
telescopes, coordinating common science goals and providing the tools to deliver science-ready
photometric and spectroscopic data. As part of this activity SPRAT~\citep{2014SPIE.9147E..8HP}
was developed as a compact, reliable, low-cost and high-throughput spectrograph and appropriate
for deployment on a wide range of 1-4\,m class telescopes. Installed on the Liverpool Telescope
since 2014, the deployable slit and grating mechanism and optical fibre based calibration
system make the instrument self-contained. Copies of SPRAT are being built for other 
telescopes. Our final task is to deliver software that can be easily configured to build
pipelines for long slit spectrographs on different telescopes.

The ``industrial standard'' of spectral and image reduction is of no doubt the
\texttt{iraf} software~\citep{1986SPIE..627..733T, 1993ASPC...52..173T}. It has powered many
reduction engines in the past and present. However, unfortunately, the software has reached
the end-of-support state where there will no longer be any official support to the software
and it relies entirely on community support\footnote{\url{https://iraf-community.github.io/}}.
In this generation of user-side Astronomy data handling and processing, as well as the
computing courses for Scientists, \texttt{Python} is among the most popular languages due to
its ease to use with a shallow learning curve, readable syntax and simple way to ``glue''
different pieces of software together. Its flexibility to serve as a scripting and an
object-oriented language makes it useful in many use cases: demonstrating with visual tools
with little overhead, prototyping, web-serving, be compiled if wanted. This broad range of
functionality and high level usage make it relatively inefficient. However, \texttt{Python}
is an excellent choice of language to build wrappers over highly efficient and well
established codes. In fact, some of the most used packages,
\texttt{scipy}~\citep{2020SciPy-NMeth} and \texttt{numpy}~\citep{2020NumPy-Array},
are written in \texttt{Fortran} and \texttt{C} respectively. Multi-threading and -processing
are also possible with built-in and other third party packages, e.g.\ mpi4py~\citep{DALCIN20111124}. 
Varies efforts are made to develop a software for the (this/)next-generation data reduction,
for example, \texttt{PypeIt}~\citep{pypeit:zenodo, pypeit:joss_pub} for taylor-made reductions
for a list of instruments, \texttt{PyReduce}~\citep{2021A&A...646A..32P} designed for optimal
Echelle spectral reduction which does not handle sky subtraction;
in the \texttt{Astropy} ``Universe''~\citep{astropy:2013, astropy:2018}, \texttt{specreduce}
is likely to be the next-generation user-focused data reduction package, but it is far from
the stage of deployment at the time of writing, \texttt{pyDIS} which has all the essential
ingredients for reducing spectra but it is out of maintenance since 2016, \texttt{specutils}
handles spectral analysis and manipulation but not reduction itself.

We use SPRAT-family as our first-light instruments for the development, but our aim is to
allow high level tools for users to build and fine tune their pipelines to support
a wide range of instruments. As of time of writing, we have successfully reduced data
from WHT/ISIS, LCO/FLOYDS, GEMINI/GMOS-LS, GTC/OSIRIS, and TNG/DOLORES; all of these are
unofficial pipelines.

By delivering near real-time data reduction we will facilitate automated or interactive
decision making, allowing "on-the-fly" modification of observing strategies and rapid
triggering of other facilities.

This paper is organised as follow, in section 2.

\section{Development and Maintenance Process}

\section{Structure of ASPIRED}

\subsection*{Image Reduction}

\subsection*{Spectral Reduction}

\subsection*{Standard Stars}

\subsection*{1D Spectrum}

\section{Spectral Reduction}

\subsection{Spectral Tracing}

\subsection{Image Rectification}

\subsection{Spectral Extraction}

\subsubsection*{Tophat Extraction}

\subsubsection*{Horne 86 Optimal Extraction}

\subsubsection*{Marsh 89 Optimal Extraction}

\subsection{Wavelength Calibration}

\subsection{Flux Calibration}

\subsubsection*{Continuum Fitting}

\subsubsection*{Smoothing}

\subsubsection*{Atmospheric Extinction Correction}

\section{Deployment}

\subsection{Liverpool Telescope/SPRAT}

\subsection{Leseide Telescope/MOKOODI}

\section{Distribution}

\section*{Acknowledgements}

The Acknowledgements section is not numbered. Here you can thank helpful
colleagues, acknowledge funding agencies, telescopes and facilities used etc.
Try to keep it short.



%%%%%%%%%%%%%%%%%%%% REFERENCES %%%%%%%%%%%%%%%%%%

% The best way to enter references is to use BibTeX:

\bibliographystyle{mnras}
\bibliography{main} % if your bibtex file is called example.bib


% Alternatively you could enter them by hand, like this:
% This method is tedious and prone to error if you have lots of references
%\begin{thebibliography}{99}
%\bibitem[\protect\citeauthoryear{Author}{2012}]{Author2012}
%Author A.~N., 2013, Journal of Improbable Astronomy, 1, 1
%\bibitem[\protect\citeauthoryear{Others}{2013}]{Others2013}
%Others S., 2012, Journal of Interesting Stuff, 17, 198
%\end{thebibliography}

%%%%%%%%%%%%%%%%%%%%%%%%%%%%%%%%%%%%%%%%%%%%%%%%%%

%%%%%%%%%%%%%%%%% APPENDICES %%%%%%%%%%%%%%%%%%%%%

\appendix

\section{Some extra material}

If you want to present additional material which would interrupt the flow of the main paper,
it can be placed in an Appendix which appears after the list of references.

%%%%%%%%%%%%%%%%%%%%%%%%%%%%%%%%%%%%%%%%%%%%%%%%%%


% Don't change these lines
\bsp	% typesetting comment
\label{lastpage}
\end{document}

% End of mnras_template.tex
